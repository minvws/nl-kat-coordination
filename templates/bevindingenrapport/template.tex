\documentclass[11pt, a4paper]{report}

\usepackage[dutch]{babel}
\usepackage{booktabs}
\usepackage{caption}
\usepackage{fancyhdr}
\usepackage{graphicx} 
\usepackage{hyperref}
\usepackage{longtable}
\usepackage[utf8]{inputenc}
\usepackage{lastpage}
\usepackage{ragged2e}
\usepackage{titlepic}
\usepackage{xcolor}


\hypersetup{
	colorlinks=true,
	linkcolor=blue,
	filecolor=magenta,      
	urlcolor=cyan,
	pdftitle={KEIKO Document by KAT},
}

\definecolor{box_color_critical}{HTML}{42145F}
\definecolor{box_color_high}{HTML}{D6293E}
\definecolor{box_color_medium}{HTML}{C36100}
\definecolor{box_color_low}{HTML}{00519C}
\definecolor{box_color_recommendation}{HTML}{C3DDF6}

\definecolor{color_critical}{HTML}{FFFFFF}
\definecolor{color_high}{HTML}{FFFFFF}
\definecolor{color_medium}{HTML}{FFFFFF}
\definecolor{color_low}{HTML}{FFFFFF}
\definecolor{color_recommendation}{HTML}{000000}

%KEIKO-specific variables
\newcommand\application{KEIKO V.0.0.2}
\newcommand\reporttitle{Bevindingenrapport voor @@{ooi.object_type}@@ @@{ooi.human_readable}@@}
\newcommand\tlp{AMBER}
\newcommand\tlpbox{\colorbox{black}{\color{orange}TLP:AMBER}}
%END-KEIKO

\pagestyle{fancy}

\fancypagestyle{plain}{
	\cfoot{\includegraphics[width=0.1\textwidth]{keiko.png}}
	\rfoot{\thepage{}\hspace{1pt} van~\pageref{LastPage}}
	\lfoot{\tlpbox}
	
	
	\renewcommand{\headrulewidth}{0pt}
	
	\chead{\includegraphics[width=0.05\textwidth]{keiko.png}}
	\lhead{\tlpbox}
	\rhead{\tlpbox}
	\renewcommand{\headrulewidth}{0pt}	
}


% Title Page
\title{ \reporttitle{} }
\author{ \application{} }
\titlepic{\includegraphics[width=70mm]{keiko.png}}

\begin{document}
\maketitle



\chapter{Over dit document}
\section{Vertrouwelijkheid}
In de informatiebeveiliging wordt gewerkt met het
\href{https://www.ncsc.nl/onderwerpen/traffic-light-protocol}{Traffic
Light Protocol (TLP)}. Dit is een internationale uniforme afspraak aan
de hand van de kleuren van het verkeerslicht. Het geeft aan hoe
vertrouwelijk informatie in het document is en of deze gedeeld mag
worden met andere personen of organisaties.

\begin{itemize}
     \item \colorbox{black}{\color{red}TLP:RED}. Deze informatie heeft
de hoogste vertrouwelijkheid. Deze mag niet met andere personen of
organisaties worden gedeeld. Vaak zal deze informatie mondeling worden
doorgegeven. In veel gevallen ook niet via e-mail of op papier, maar het
kan natuurlijk wel.
     \item \colorbox{black}{\color{orange}TLP:AMBER}. Deze informatie
mag op een need to know-basis worden gedeeld binnen de eigen organisatie
en de klanten (of aangesloten partijen).
     \item \colorbox{black}{\color{orange}TLP:AMBER+STRICT}. Deze
informatie mag alleen binnen de eigen organisatie worden gedeeld met
mensen voor wie toegang noodzakelijk is. Dit is op een `need to
know'-basis binnen de eigen organisatie.
     \item \colorbox{black}{\color{green}TLP:GREEN}. Deze informatie is
beschikbaar voor iedereen binnen de gemeenschap, waarop ze gericht is.
Dat betekent dat het nuttig kan zijn en daarmee gedeeld kan worden op
basis van `nice to know'. Er is geen restrictie tot de eigen organisatie.
     \item \colorbox{black}{\color{white}TLP:CLEAR}. Deze informatie is
niet vertrouwelijk en kan openbaar worden gedeeld.
\end{itemize}

\textbf{Dit document is gerubriceerd als \underline{TLP:\tlp}.}


\tableofcontents

\newpage

\chapter{Overzicht}

\section{Samenvatting}
Dit zijn de bevindingen van een OpenKAT-analyse op @@{ valid_time.astimezone().strftime("%Y-%m-%d %H:%m:%S %Z") }@@. % chktex 36 chktex 18

\bgroup{}
\def\arraystretch{1.2}
\section{Totalen}
\begin{tabular}{ llr }
	Niveau & Uniek & Totaal aantal voorvallen \\\toprule
	\toprule
	
		\colorbox{box_color_@@{ level_name }@@}{ \color{color_@@{ level_name }@@} @@{ level_name }@@ } & @@{ meta.total_by_severity_per_finding_type[level_name] }@@ & @@{ level_sum }@@ \\
	
	\bottomrule
	Totaal & @@{meta.total_finding_types}@@ & @@{meta.total}@@
\end{tabular}
\egroup{}

\bgroup{}
\def\arraystretch{1.2}
\section{Bevinding types}
\begin{tabular}{ llr }
	Risico niveau & Bevindingstype & Voorvallen \\\toprule
	\midrule
	
		\colorbox{box_color_@@{ occurrence.finding_type.risk_level_severity }@@}{ \color{color_@@{ occurrence.finding_type.risk_level_severity }@@} @@{ occurrence.finding_type.risk_level_severity }@@ } & @@{ occurrence.finding_type.id }@@ & @@{occurrence.list|length}@@ \\
	
	\bottomrule
\end{tabular}
\egroup{}


\chapter{Bevindingen}

	\section{@@{finding_type_id}@@}
	\subsection{Bevinding informatie}
	\begin{longtable}{ p{.25\textwidth}  p{.75\textwidth} }
		Bevinding & @@{occurrence.finding_type.id}@@ \\
		Risico niveau & @@{occurrence.finding_type.risk_level_score}@@ / 10 \\
		
			CVSS & @@{occurrence.finding_type.cvss}@@ \\
		
		Ernst & @@{occurrence.finding_type.risk_level_severity|capitalize}@@ \\
		Beschrijving & @@{occurrence.finding_type.description}@@ \\
		
			Informatie & @@{occurrence.finding_type.Information}@@ \\
		
		
			Aanbeveling & @@{occurrence.finding_type.recommendation}@@ \\
		
		
			Bron& \href{@@{occurrence.finding_type.source}@@}{@@{occurrence.finding_type.source}@@} \\
		
		
			Informatie laatst bijgewerkt & @@{occurrence.finding_type.information_updated}@@ \\
		
	\end{longtable}

	\subsection{Voorvallen}
	
		\subsubsection{@@{finding.ooi.human_readable}@@}
		@@{finding.description}@@
	




\chapter{Verklarende Woordenlijst}
\begin{longtable}{ p{.25\textwidth}  p{.75\textwidth} } \toprule
	\textbf{Begrip} & \textbf{Betekenis} \\\toprule \endhead{}
	
		@@{ term }@@ & @@{ description }@@ \\ \midrule
	
	\bottomrule
\end{longtable}

\end{document}          
